\documentclass[twocolumn,amssymb,prb,aps,superscriptaddress]{revtex4}

\usepackage{graphicx}
\usepackage[utf8]{inputenc}
\usepackage{amsmath}
\usepackage{float}


\begin{document}

\begin{abstract}
    Hola Abstract, este es un pequeño resumen del documento presentado
\end{abstract}

\title{Resonancia Paramagnetica Spin Electrón}
\author{Majo}

\affiliation{Colegio de Ciencias e Ingeniería, USFQ, Quito, Ecuador} 

\author{Martin}

\affiliation{Colegio de Ciencias e Ingeniería, USFQ, Quito, Ecuador}

\date{\today}

\maketitle

\section[Intro]{Introducción}
\label{sec:intro}

\section[]{Cuántica del Sistema}
\label{sec:cuantica}

\subsection*{Momentos Angulares y Spines}
\label{sec:momentosAngulares}

\subsection*{Estadistica del sistema}
\label{mecanicaEstadistica}

    \begin{figure}[H]
        \centering
        \includegraphics*[width=6.0cm]{Images/EPR_splitting}
        \caption{Diagrama de diferencia de Energia}
        \label{fig:diagramaDiferenciaDeEnergia} 
    \end{figure}

\section[]{Aplicación 1}
\label{sec:aplicacion1}

\section[]{Aplicación 2}
\label{sec:aplicacion2}

\end{document}